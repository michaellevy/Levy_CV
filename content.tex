\section{Education}\label{education}

\begin{description}
\tightlist
\item[2017 (anticipated)]
PhD in Ecology: Environmental Policy and Human Ecology, University of
California, Davis
\item[2012]
MS in Biology, West Virginia University
\item[2004]
BS in Chemistry, Fort Lewis College
\end{description}

\section{Publications}\label{publications}

\subsection{Peer-Reviewed
Publications}\label{peer-reviewed-publications}

\begin{description}
\tightlist
\item[In review]
Mark Lubell, \textbf{Michael Levy}, Aseem Prakash, and Matt Potoski.
Clubs Old and New: Social Networks and Participation in Voluntary
Environmental Programs.
\item[2014]
\textbf{Michael Levy} and Jonathan Cumming. Development of soils and
communities of plants and arbuscular mycorrhizal fungi on West Virginia
surface mines. Environmental Management 54(5): 1153-1162.
\end{description}

\subsection{Non-Refereed Publications}\label{non-refereed-publications}

\begin{description}
\tightlist
\item[2015]
\textbf{Michael Levy} and Matthew B. Hoffman. Lodi rules for
stustainable winegrowing certification program, 2015 report.
\item[2013]
\textbf{Michael Levy}. Integrating science, management, and policy for
the future of water in California. Eos, 94(29): 256
\item[2013]
\textbf{Michael Levy}, Matthew B. Hoffman, A. Vicken Hillis, and Mark N.
Lubell. Sustainability certification: Why do vineyards participate?
Policy Brief for UC Davis Policy Institute.
\item[2013]
\textbf{Michael Levy}. Measure I is our best option for necessary water
improvements. The Davis Enterprise, February 10, 2013: A15
\item[2010 - 2012]
\textbf{Michael Levy}. Weekly environmental science and policy opinion
column. The Daily Athenaeum, 124(12) - 125(94)
\end{description}

\section{Presentations}\label{presentations}

\subsection{Invited Talks}\label{invited-talks}

\begin{description}
\tightlist
\item[2015]
\textbf{Michael Levy}. Predicting interdependence: Inference from
exponential random graph models of social networks. Statistical
Methodology in the Social Sciences Conference, UC Davis Institute for
Social Sciences, Davis CA
\item[2015]
\textbf{Michael Levy}. Exponential random graph models for statistical
inference on networks. Statistical Sciences Symposium, UC Davis
Department of Statistics, Davis CA
\item[2014]
\textbf{Michael Levy}. \texttt{dplyr}: Data manipulation in R made easy.
Davis R-Users' Group, Davis CA
\end{description}

\subsection{Professional Conferences}\label{professional-conferences}

\begin{description}
\tightlist
\item[2015]
\textbf{Michael Levy} and Mark Lubell. Innovation and cooperation in the
structure of agriculture information networks. American Political
Science Association Annual Meeting, San Francisco CA
\item[2015]
\textbf{Michael Levy} and Mark Lubell. Innovation and cooperation in
agricultural information networks. Political Networks Conference,
Portland, OR
\item[2015]
\textbf{Michael Levy}. Cultural multilevel selection and social networks
in sustainable viticulture. Biennial Global Conference: International
Association for the Study of the Commons, Edmonton AB, Canada
\item[2014]
\textbf{Michael Levy}. Cultural diffusion and polarization: An
agent-based model. Center for Advanced Modeling Graduate Workshop,
Baltimore MD
\item[2014]
\textbf{Michael Levy}, Matthew B. Hoffman, A. Vicken Hillis, and Mark N.
Lubell. Who joins the club? Learning pathways for sustainable
agriculture certification. Environmental Politics Conference, Santa
Barbara CA
\item[2014]
\textbf{Michael Levy}, Matthew B. Hoffman, A. Vicken Hillis, and Mark N.
Lubell. Sustainable agriculture certification: Who's joining?
Interdisciplinary Graduate and Professional Symposium, Davis CA
\item[2011]
\textbf{Michael Levy} and Jonathan R. Cumming. Development of soil,
biodiversity, and arbuscular mycorrhizae on pasture-reclaimed surface
mines in Appalachia. Annual Meeting of the Ecological Society of
America, Austin TX
\item[2011]
\textbf{Michael Levy} and Jonathan R. Cumming. What happens after the
mining is done? Development of soils and biodiversity on reclaimed
surface mines. Science and Technology in Society, Morgantown WV
\end{description}

\subsection{Outreach Presentations}\label{outreach-presentations}

\begin{description}
\tightlist
\item[2014]
\textbf{Michael Levy}. Environmental certifications among
viticulturists. Lodi Winegrape Commission, Davis CA
\item[2012]
\textbf{Michael Levy}. Stripping wild and wonderful: Truths and myths
about surface mining in West Virginia. Science on Tap public science
forum, Morgantown WV
\item[2011]
\textbf{Michael Levy} and Terence Houston. Weighing the long-term
implications of mountaintop removal mining. Ecological Society of
America podcast, The Ecologist Goes To Washington
\item[2011]
Bob H. Baber, Ted Boettner, \textbf{Michael Levy}, and Jeff G. Skousen.
Coal: The lifeblood of West Virginia? Democracy Matters panel
discussion, Morgantown WV
\end{description}

\section{Awards}\label{awards}

\begin{description}
\tightlist
\item[2015]
Professors for the Future Fellowship, UC Davis
\item[2014]
Fire on the Fringe, SESYNC Graduate Student Pursuit (co-PI)
\item[2014]
Sustainable Agriculture Scholarship, Annie's Homegrown
\item[2014]
Henry A. Jastro Research Award, UC Davis College of Agricultural and
Environmental Sciences
\item[2012]
Climate Change, Water, and Society, Integrative Graduate Education
Research Traineeship, National Science Foundation
\item[2012]
Ecology Fellowship, UC Davis Graduate Group in Ecology
\item[2011]
Graduate Student Policy Award, Ecological Society of America
\item[2009]
HERF Supplementary Fellowship, West Virginia University
\item[2004]
Senior Award, Fort Lewis College, Department of Chemistry
\item[2003]
Callendar Chemistry Scholarship, Fort Lewis College
\item[2003]
Research Experience for Undergraduates, National Science Foundation
\end{description}

\section{Teaching Experience}\label{teaching-experience}

\subsection{University of California,
Davis}\label{university-of-california-davis}

\begin{description}
\tightlist
\item[2015]
Instructor, R Bootcamp. Developed and taught one-week, 30-classroom hour
intensive introduction to the statistical programming language
\texttt{R}. Attended by 200 graduate students and post-docs
\item[2015]
Co-Instructor, Data Carpentry Workshop
\item[2013]
Organizer and Lead, Seminar on agent-based models for the study of
coupled human and natural systems
\end{description}

\subsection{Interuniversity Consortium for Political and Social Research
(ICPSR)}\label{interuniversity-consortium-for-political-and-social-research-icpsr}

\begin{description}
\tightlist
\item[2015]
Teaching Assistant, Introduction to Social Network Analysis in R
\end{description}

\subsection{West Virginia University}\label{west-virginia-university}

\begin{description}
\tightlist
\item[2011]
Teaching Assistant, Environmental Biology Laboratory
\item[2010 - 2011]
Teaching Assistant, Biology Capstone Course: Total Science Experience
\item[2009 - 2010]
Teaching Assistant, Principles of Biology Laboratory
\end{description}

\subsection{Other}\label{other}

\begin{description}
\tightlist
\item[2008 - 2009]
English as a Foreign Language Teacher, Dogye Elementary School,
Gangwon-do South Korea
\item[2007 - 2008]
MCAT Instructor, The Princeton Review, Boulder CO
\item[2005 - 2006]
English as a Foreign Language Teacher, Han Yun Sik English Academy,
Busan South Korea
\end{description}

\subsection{Fort Lewis College}\label{fort-lewis-college}

\begin{description}
\tightlist
\item[2004]
Teaching Assistant, Organic Chemistry Laboratory
\item[2003]
Teaching Assistant, Analytical Chemistry Laboratory
\item[2001 - 2003]
Teaching Assistant, General Chemistry Laboratory
\end{description}

\section{Additional Education}\label{additional-education}

\begin{description}
\tightlist
\item[2015]
Instructor training. Software Carpentry, Davis CA
\item[2014]
Social Media and Socio-Environmental Systems. SESYNC workshop, Annapolis
MD
\item[2013]
Multi-platform International Summer School on Agent-Based Modeling and
Simulation. Cirad, Montpellier France
\item[2011]
Becoming the Messenger. Science communication training, National Science
Foundation, Morgantown WV
\item[2011]
Lobbying Congress for Science, Biological Ecological Sciences Coalition,
Washington DC
\end{description}

\section{Service}\label{service}

\subsection{UC Davis}\label{uc-davis}

\begin{description}
\tightlist
\item[2015 -]
Mentor, Dustan Li Undergraduate Research Project
\item[2015 -]
Co-administrator, Davis R Users' Group
\item[2013 -]
Member, Admissions Committee, Graduate Group in Ecology
\item[2015]
Founder, Interdisciplinary Network Science Group
\item[2015]
Graduate Student Representative, Climate Policy Faculty Search
Committee, Department of Environmental Science and Policy
\item[2013]
Mentor, AggieMentors high school student research program
\item[2013]
Organizer, Future of Water in California Conference
\item[2013]
Delegate, UC Day in DC
\end{description}

\subsection{West Virginia University}\label{west-virginia-university-1}

\begin{description}
\tightlist
\item[2010 - 2011]
Mentor, Kylen Whipp Undergraduate Honors Thesis
\end{description}

\subsection{Fort Lewis College}\label{fort-lewis-college-1}

\begin{description}
\tightlist
\item[2003 - 2004]
President, Chemistry Club
\item[2002 - 2004]
Co-Founder, Green Chemistry
\item[2002 - 2004]
President and Vice President, Uniting Students through Wellness
\end{description}

\begin{center}\rule{0.5\linewidth}{\linethickness}\end{center}

\section{Teaching statement}\label{teaching-statement}

\subsection{Teaching philosophy}\label{teaching-philosophy}

Teaching is a performance art, and my first priority is holding my
students' attention. Enthusiasm goes a long way, and I bring the joy and
satisfaction that computing and statistics bring me into the classroom.
Also critical is showing students the payoff of what they are learning,
early and often. This motivates struggling through challenging details,
which is especially important for detail-heavy methods courses. Even
better than showing students the payoff is giving them an experiential
taste of it. \medskip

I start my introduction to scientific computing classes with advanced
visualization -- within the first class session students are creating
beautiful, informative plots. Of course, they don't understand the
details of the code that generated the plots, but they can play with it,
and this connection to the material sustains motivation to wrestle with
details in the weeks to come. \medskip

I subscribe to the philosophy of student-centered learning and the
strategy of backward course design. This means that I start planning a
course by defining the students' learning objectives in the context of
their degree program; then I design summative assessments, completing
which demonstrates meeting the objectives; and lastly I design lectures,
activities, and formative assessments to get students to the point where
they can complete the summative assessments and so reach the course
objectives. I structure a lot of interactivity in the classroom, both
among students and between myself and students, and I also ask for
regular feedback, for example, using a quick Google Form to have
students rate their understanding of a week's material, which I might
quickly plot and show the students. This interaction helps keep me and
the students aware of how well everyone is doing. That gives the
students a better sense of how they might need to adjust their effort or
when it would be beneficial to others to interrupt the class to ask a
question, and it allows me to adjust my pace and methods to student
needs. \medskip

I build flexibility into my methods courses by writing code in front of
the class in every lecture. This might seem like an odd approach, but it
has significant advantages and has worked very well for myself and
others. It provides integrated flexibility; it forces slower delivery
which tends to align well with the rate at which students absorb
information; it shows the process of doing scientific computing rather
than just the product; it demonstrates that I make mistakes and how I
diagnose and fix them; and it allows me to demonstrate how I use the
computer to figure out tough concepts. I generally deliver the code I
write to students in real-time through a web link so that they can try
to code along with me but can catch up by referencing my code if they
miss something. They can then switch seamlessly from following along
with my code to writing their own code during in-class exercises, which
I use extensively.

\subsection{Teaching experience}\label{teaching-experience-1}

I have taught a wide variety of subjects across all student levels.
During my doctoral studies, I served as a Teaching Assistant for a
PhD-level Bayesian statistics class for which the professor was out of
the country for the duration of course (lectures were pre-recorded and I
was the primary contact for the XX students). As part of my Professors
for the Future Fellowship, I developed a one-week, 30-classroom-hour
intensive introduction to the statistical programming language
\texttt{R}. The course was made available to graduate students and
post-docs at UC Davis, over 350 of whom registered. I delivered the
class to 200 students, and it was widely well reviewed (Mean student
evaluations: instructor's overall teaching: 4.6/5, inspired and
motivated student interest: 4.7/5, demonstrated concern for student
learning: 4.9/5). I also TA-ed for the Interuniversity Consortium for
Political and Social Research (ICPSR) during the summers of my PhD
program. \medskip

During my master's, I taught a capstone Biology course in which groups
of undergraduates executed a research project from conceptualization and
proposal to analysis and presentation in a semester, and I TA-ed
introductory and environmental biology courses. I TA-ed chemistry
laboratories as an undergraduate, and between my undergraduate and
graduate programs, I taught English in South Korea and Medical College
Admission Test (MCAT) preparation for the Princeton Review.

I have also mentored many student research projects. In addition to the
above-mentioned research class, I mentored an undergraduate's honor's
thesis on the effects of coal ash application to surface mines during my
master's. I also incorporated an undergraduate into my doctoral research
and mentored high school students' research projects through the
EnviroMentors program.

\subsection{Teaching interests}\label{teaching-interests}

I am primarily interested in teaching applied methodology courses at a
variety of levels. I enjoy showing students how a computational or
statistical technique can provide insight into a substantive question of
interest. I can teach undergraduate and graduate statistics and can
incorporate computation into those classes in such a way that builds
that skill as well as bolstering statistical understanding. I would love
to teach advanced classes in network analysis and agent-based modeling.
I could also teach computing classes largely separate statistics,
covering topics such as version control, data and code management, and
automation through shell scripting. Substantively, I am qualified to
teach classes on individual decision making, policy analysis, and
social-ecological systems.
